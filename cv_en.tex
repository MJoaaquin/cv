%%%%%%%%%%%%%%%%%%%%%%%%%%%%%%%%%%%%%%%%%
% "ModernCV" CV and Cover Letter
% LaTeX Template
% Version 1.11 (19/6/14)
%
% This template has been downloaded from:
% http://www.LaTeXTemplates.com
%
% Original author:
% Xavier Danaux (xdanaux@gmail.com)
%
% License:
% CC BY-NC-SA 3.0 (http://creativecommons.org/licenses/by-nc-sa/3.0/)
%
% Important note:
% This template requires the moderncv.cls and .sty files to be in the same 
% directory as this .tex file. These files provide the resume style and themes 
% used for structuring the document.
%
%%%%%%%%%%%%%%%%%%%%%%%%%%%%%%%%%%%%%%%%%

%----------------------------------------------------------------------------------------
%	PACKAGES AND OTHER DOCUMENT CONFIGURATIONS
%----------------------------------------------------------------------------------------

\documentclass[11pt,a4paper,sans]{moderncv} % Font sizes: 10, 11, or 12; paper sizes: a4paper, letterpaper, a5paper, legalpaper, executivepaper or landscape; font families: sans or roman

\moderncvstyle{classic} % CV theme - options include: 'casual' (default), 'classic', 'oldstyle' and 'banking'
\moderncvcolor{green} % CV color - options include: 'blue' (default), 'orange', 'green', 'red', 'purple', 'grey' and 'black'

\usepackage{lipsum} % Used for inserting dummy 'Lorem ipsum' text into the template

\usepackage[scale=0.75]{geometry} % Reduce document margins
%\setlength{\hintscolumnwidth}{3cm} % Uncomment to change the width of the dates column
%\setlength{\makecvtitlenamewidth}{10cm} % For the 'classic' style, uncomment to adjust the width of the space allocated to your name

\setlength{\makecvtitlenamewidth}{4cm}
%----------------------------------------------------------------------------------------
%	NAME AND CONTACT INFORMATION SECTION
%----------------------------------------------------------------------------------------

\firstname{Joaquin} % Your first name
\familyname{Mansilla} % Your last name

% All information in this block is optional, comment out any lines you don't need
\title{Curriculum Vitae}
\address{}{Villa La Angostura, Neuqu\'{e}n, Argentina}
\email{	joaquin.mansilla@lawal.coop}
%\extrainfo{additional information}
\photo[70pt][0.4pt]{pictures/picture} % The first bracket is the picture height, the second is the thickness of the frame around the picture (0pt for no frame)

%----------------------------------------------------------------------------------------

\begin{document}

\makecvtitle % Print the CV title

%----------------------------------------------------------------------------------------
%	EDUCATION SECTION
%----------------------------------------------------------------------------------------

\section{Education}

\cventry{2011--2015}{High School Degree with orientation in Accounting}{High School}{}{}{Centro Provincial de Ense\~{n}anza Media 17 (Villa La Angostura, Neuqu\'{e}n)}

\subsection{Courses}
\cventry{2016}{Introduction to Programming}{Consultora Lehrart}{}{}{}
\cventry{2016}{Game Development Design}{Universidad Tecnologica Nacional}{}{}{}
\cventry{2018}{Full Stack Developer}{Universidad Tecnologica Nacional}{}{}{}
\cventry{2018}{Responsive Web Desing}{Free Code Camp}{}{}{with the mentoring of Fiqus}
\cventry{2018}{JavaScript Algorithms and Data Structures}{Free Code Camp}{}{}{with the mentoring of Fiqus}
\cventry{2020}{The Complete Elixir and Phoenix Bootcamp}{Udemy}{}{}{}
\cventry{2020}{Functional Programming in Erlang}{Future Learn}{}{}{}
\cventry{2020}{Concurrent Programming in Erlang}{Future Learn}{}{}{}
\cventry{2024}{Ethical Hacking}{Universidad Tecnologica Nacional}{}{}{}

%----------------------------------------------------------------------------------------
%	WORK EXPERIENCE SECTION
%----------------------------------------------------------------------------------------

\section{Experience}

%------------------------------------------------
\cventry{2018-2020}{Front End Developer}{\textsc{Betterez}}{Canada}{}{I worked for a Canadian company in the transport sector, programming in NodeJS and in the last months in the project also I worked in Elixir. One of the main objectives of this client was to migrate a monolithic application to different micro-services to get a network of to obtain a network of APIs and REST services
\newline{}\newline{}
Technologies used:
\newline{} NodeJS, Express, Vue, JQuery, Webpack, Elixir, Phoenix, Ecto, MongoDB, PostgreSQL.
}
%------------------------------------------------

%------------------------------------------------
\cventry{2020}{Full Stack Developer - DevOps}{\textsc{Emprendedores}}{Argentina}{}{Implementation of the free software prestashop in a catalogue so that small producers of villa la angostura can sell their products during the quarantine
\newline{}\newline{}
Technologies used:
\newline{} Prestashop, Docker.
}
%------------------------------------------------

%------------------------------------------------
\cventry{2020-2021}{Full Stack Developer}{\textsc{Buenbit}}{Argentina}{}{In buenbit I was working mainly in backend, for the user compliance department. I also participated in the development of the mobile application in the initial layout and in the development of features
\newline{}\newline{}
Technologies used:
\newline{} Django, ReactJS, React Native, Ruby, PostgreSQL, Docker, Kubernetes.
}
%------------------------------------------------

%------------------------------------------------
\cventry{2022}{Full Stack Developer - DevOps}{\textsc{Origino}}{Argentina}{}{Development of a billing system and a web application.
\newline{}\newline{}
Technologies used:
\newline{} Django, ReactJS, PostgreSQL, Docker, GCloud.
}
%------------------------------------------------

%------------------------------------------------
\cventry{2022-Present}{Full Stack Developer}{\textsc{Cedalio}}{Argentina}{}{Developed a CLI in Rust to interact with smart contracts. Built a web application with Rust that persists data on a smart contract. Developed a web application in Elixir.
\newline{}\newline{}
Technologies used:
\newline{} Elixir, Rust, Typescript, Web3, PostgreSQL, Graphql, AWS.
}
%------------------------------------------------

\subsection{Cooperative Open Source Projects}

%------------------------------------------------
\cventry{2020-Present}{Full Stack Developer - DevOps}{\textsc{Archivos del Sur}}{Argentina}{}{Application for mapping historical settlements located in the Nahuel Huapi and surroundings
\newline{}\newline{}
Technologies used:
\newline{} Omeka, Docker
}
%------------------------------------------------

%------------------------------------------------
\cventry{2020-Present}{Full Stack Developer - DevOps}{\textsc{Pirra}}{Fiqus Labs}{}{ Implementation of an electronic invoicing system for educational use in schools in Villa a Angostura
\newline{}\newline{}
Technologies used:
\newline{} Django, PostgreSQL, Docker, Nginx.
}
%------------------------------------------------

%------------------------------------------------
\cventry{2018-2019}{Full Stack Developer}{\textsc{Surgex}}{Fiqus Labs}{}{An open sourced web application done as a donation with heartbeat from Fiqus to the surgery services of Argentina. https://github.com/fiqus/surgex
\newline{}\newline{}
Technologies used:
\newline{} Elixir, Phoenix, Ecto, Vue, PostgreSQL.
}
%------------------------------------------------

%------------------------------------------------
\cventry{2019}{Full Stack Developer}{\textsc{Coophub}}{Fiqus Labs}{}{This web app uses the GitHub API and GitLab API to fetch, process and nicely display the projects/repositories of any subscribed cooperative from over the world. https://coophub.io/
\newline{}\newline{}
Technologies used:
\newline{} Elixir, Phoenix, Cachex, ReactJS.
}
%------------------------------------------------


%------------------------------------------------
\cventry{2021}{Front End Developer}{\textsc{Compras Comunitarias}}{Fiqus Labs}{}{It is a platform for ordering in community purchasing. We are currently starting to migrate the django project to a python api with a react front end.
\newline{}\newline{}
Technologies used:
\newline{} Django, PostgreSQL, FastApi, ReactJS.
}
%------------------------------------------------

\pagebreak

\subsection{SpawnFest}

%------------------------------------------------
\cventry{2019}{Front End Developer}{\textsc{Prexent}}{}{}{Fast, live and beautiful presentations from Markdown powered by Phoenix LiveView. Winners of Usefulness and Completion categories https://github.com/fiqus/prexent
\newline{}\newline{}
Technologies used:
\newline{} Elixir, Phoenix, LiveView.
}
%------------------------------------------------

%------------------------------------------------
\cventry{2020}{Front End Developer}{\textsc{Arena LiveView}}{}{}{ LiveView implementation of the 3d-css-scene package, integrated with Phoenix Sockets and Presence to track and broacast connections and movement in a 3D room
\newline{}\newline{}
Technologies used:
\newline{} Elixir, Phoenix, LiveView.
}
%------------------------------------------------


%----------------------------------------------------------------------------------------
%	COMPUTER SKILLS SECTION
%----------------------------------------------------------------------------------------

%\section{Computer skills}
%
%\cvitemwithcomment{Intermediate}{Javascript}{NodeJS, Express, VueJS, JQuery}
%\cvitemwithcomment{}{Databases}{MongoDB, PostgreSQL}
%
%\cvitemwithcomment{Intermediate}{Elixir}{Phoenix, Ecto, LiveView, Cachex}
%\cvitemwithcomment{Basic}{Erlang}{Standard Libraries}

%----------------------------------------------------------------------------------------
%	LANGUAGES SECTION
%----------------------------------------------------------------------------------------

\section{Languages}

\cvitemwithcomment{Spanish}{Mothertongue}{}
\cvitemwithcomment{English}{B1}{}

%----------------------------------------------------------------------------------------
%	INTERESTS SECTION
%----------------------------------------------------------------------------------------


%----------------------------------------------------------------------------------------
%	COVER LETTER
%----------------------------------------------------------------------------------------

% To remove the cover letter, comment out this entire block

%\clearpage
%
%\recipient{HR Department}{Corporation\\123 Pleasant Lane\\12345 City, State} % Letter recipient
%\date{\today} % Letter date
%\opening{Dear Sir or Madam,} % Opening greeting
%\closing{Sincerely yours,} % Closing phrase
%\enclosure[Attached]{curriculum vit\ae{}} % List of enclosed documents
%
%\makelettertitle % Print letter title
%
%\lipsum[1-3] % Dummy text
%
%\makeletterclosing % Print letter signature

%----------------------------------------------------------------------------------------

\end{document}